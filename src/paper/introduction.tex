
\documentclass[12pt, a4paper, oneside]{scrartcl}

\usepackage[paper=a4paper,left=3cm,right=2cm,top=2cm,bottom=2cm,footskip=24pt]{geometry}



\title{}


\begin{document}


\maketitle
\section{Introduction}

The CoViD-19 pandemic was officially declared a Public Health Emergency of International Concern on January 30, 2020 and a global pandemic on March 11, 2020 by the World Health Organization (WHO). The presence of extensive literature regarding this global pandemic has led us to understand the unexpected nature of this pandemic and has motivated further investigation into the various consequences of this pandemic. The  CoViD-19 pandemic affects societies at the core, causing human suffering and promoting disconnectedness amongst members of a social group. To be able to efficiently combat this virus public health officials have encouraged the practice of strict social distancing and participating in quarantine measures. With the implementation of strict policies to restrict the spread of this virus we have overtime observed a reduction in contact amongst individuals. The promotion of such policies has been found to be correlated with the number of cases in the economy, with higher number of cases, government and public health officials implement much stricter social distancing policies, and once the cases begin to decrease these policies have been observed to become lenient. The most efficient form of measuring the frequency of a certain event, such as in our case CoViD-19 pandemic, occurs over a specified period of time and has been therefore characterized as the most efficient indicator of improvement amongst the population in terms of this pandemic. For these reasons, in our study we attempt to provide an in depth analysis of  the association between the incidence rate in the Netherlands and contact reduction behavior amongst the population using LISS (Longitudinal Internet Studies for the Social Sciences) panel data. The LISS panel data is an internet based household panel administered by the centERdata (Tilburg University, The Netherlands). The LISS panel data is an internet based household panel administered by the centERdata (Tilburg University, The Netherlands).  The questionnaires have been conducted to survey participants of the LISS panel with regard to their experiences, expectations and their socioeconomic situation at the time of the 2020 coronavirus pandemic. The LISS panel data provides extensive surveys for each of the waves, with waves consisting from wave one to wave seven.   To be able to attain further information exclusive to contact reduction we have limited our focus on variables directly related to contact reduction behavior of individuals from wave one to wave five present in the LISS panel data.  Information related to the Netherlands including the incidence rate (for new cases and total cases) has been collected from Our World in Data, a scientific online publication focused on large global issues. 

In this paper we study the association between the incidence rate in the Netherlands and contact reduction amongst the population mostly through graphical representation. Unfortunately, we have not been able to attain relevant variables which are common in all the waves, apart from three contact reduction variables in wave four and five, which resulted in the process of obtaining a precise estimation of the effect of the incidence rate on contact reduction quite cumbersome. For these reasons we do not graphically present a time series of contact reduction with the incidence rate over a specified period of time, we however do present an efficient representation of reduction in contact with an increase in the incidence rate for new cases in the Netherlands. We further contribute to literature by applying an ordinary least squares statistical method to study the association between the average monthly incidence rate with respect to the contact reduction variables for a specific month, in addition we continue with performing the OLS by taking the logarithm of the incidence rate to  show the proportional reduction in contact overtime. 



\end{document}
